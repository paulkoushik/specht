% generated by GAPDoc2LaTeX from XML source (Frank Luebeck)
\documentclass[a4paper,11pt]{report}

\usepackage[top=37mm,bottom=37mm,left=27mm,right=27mm]{geometry}
\sloppy
\pagestyle{myheadings}
\usepackage{amssymb}
\usepackage[latin1]{inputenc}
\usepackage{makeidx}
\makeindex
\usepackage{color}
\definecolor{FireBrick}{rgb}{0.5812,0.0074,0.0083}
\definecolor{RoyalBlue}{rgb}{0.0236,0.0894,0.6179}
\definecolor{RoyalGreen}{rgb}{0.0236,0.6179,0.0894}
\definecolor{RoyalRed}{rgb}{0.6179,0.0236,0.0894}
\definecolor{LightBlue}{rgb}{0.8544,0.9511,1.0000}
\definecolor{Black}{rgb}{0.0,0.0,0.0}

\definecolor{linkColor}{rgb}{0.0,0.0,0.554}
\definecolor{citeColor}{rgb}{0.0,0.0,0.554}
\definecolor{fileColor}{rgb}{0.0,0.0,0.554}
\definecolor{urlColor}{rgb}{0.0,0.0,0.554}
\definecolor{promptColor}{rgb}{0.0,0.0,0.589}
\definecolor{brkpromptColor}{rgb}{0.589,0.0,0.0}
\definecolor{gapinputColor}{rgb}{0.589,0.0,0.0}
\definecolor{gapoutputColor}{rgb}{0.0,0.0,0.0}

%%  for a long time these were red and blue by default,
%%  now black, but keep variables to overwrite
\definecolor{FuncColor}{rgb}{0.0,0.0,0.0}
%% strange name because of pdflatex bug:
\definecolor{Chapter }{rgb}{0.0,0.0,0.0}
\definecolor{DarkOlive}{rgb}{0.1047,0.2412,0.0064}


\usepackage{fancyvrb}

\usepackage{mathptmx,helvet}
\usepackage[T1]{fontenc}
\usepackage{textcomp}


\usepackage[
            pdftex=true,
            bookmarks=true,        
            a4paper=true,
            pdftitle={Written with GAPDoc},
            pdfcreator={LaTeX with hyperref package / GAPDoc},
            colorlinks=true,
            backref=page,
            breaklinks=true,
            linkcolor=linkColor,
            citecolor=citeColor,
            filecolor=fileColor,
            urlcolor=urlColor,
            pdfpagemode={UseNone}, 
           ]{hyperref}

\newcommand{\maintitlesize}{\fontsize{50}{55}\selectfont}

% write page numbers to a .pnr log file for online help
\newwrite\pagenrlog
\immediate\openout\pagenrlog =\jobname.pnr
\immediate\write\pagenrlog{PAGENRS := [}
\newcommand{\logpage}[1]{\protect\write\pagenrlog{#1, \thepage,}}
%% were never documented, give conflicts with some additional packages

\newcommand{\GAP}{\textsf{GAP}}

%% nicer description environments, allows long labels
\usepackage{enumitem}
\setdescription{style=nextline}

%% depth of toc
\setcounter{tocdepth}{1}





%% command for ColorPrompt style examples
\newcommand{\gapprompt}[1]{\color{promptColor}{\bfseries #1}}
\newcommand{\gapbrkprompt}[1]{\color{brkpromptColor}{\bfseries #1}}
\newcommand{\gapinput}[1]{\color{gapinputColor}{#1}}


\begin{document}

\logpage{[ 0, 0, 0 ]}
\begin{titlepage}
\mbox{}\vfill

\begin{center}{\maintitlesize \textbf{\textsf{Circle}\mbox{}}}\\
\vfill

\hypersetup{pdftitle=\textsf{Circle}}
\markright{\scriptsize \mbox{}\hfill \textsf{Circle} \hfill\mbox{}}
{\Huge \textbf{Adjoint groups of finite rings\mbox{}}}\\
\vfill

{\Huge Version 1.0\mbox{}}\\[1cm]
{25 October 2024\mbox{}}\\[1cm]
\mbox{}\\[2cm]
{\Large \textbf{Alexander Konovalov    \mbox{}}}\\
{\Large \textbf{Panagiotis Soules   \mbox{}}}\\
\hypersetup{pdfauthor=Alexander Konovalov    ; Panagiotis Soules   }
\end{center}\vfill

\mbox{}\\
{\mbox{}\\
\small \noindent \textbf{Alexander Konovalov    }  Email: \href{mailto://alexk@mcs.st-andrews.ac.uk} {\texttt{alexk@mcs.st-andrews.ac.uk}}\\
  Homepage: \href{http://www.cs.st-andrews.ac.uk/~alexk/} {\texttt{http://www.cs.st-andrews.ac.uk/\texttt{\symbol{126}}alexk/}}\\
  Address: \begin{minipage}[t]{8cm}\noindent
 School of Computer Science\\
 University of St Andrews\\
 Jack Cole Building, North Haugh,\\
 St Andrews, Fife, KY16 9SX, Scotland \end{minipage}
}\\
{\mbox{}\\
\small \noindent \textbf{Panagiotis Soules   }  Email: \href{mailto://psoules@math.uoa.gr} {\texttt{psoules@math.uoa.gr}}\\
  Address: \begin{minipage}[t]{8cm}\noindent
 Department of Mathematics\\
 National and Capodistrian University of Athens\\
 Panepistimioupolis, GR-15784, Athens, Greece \end{minipage}
}\\
\end{titlepage}

\newpage\setcounter{page}{2}
{\small 
\section*{Abstract}
\logpage{[ 0, 0, 1 ]}
 \index{Circle package@\textsf{Circle} package} The \textsf{GAP}4 package \textsf{Circle} extends the \textsf{GAP} functionality for computations in adjoint groups of associative rings. It
provides functionality to construct circle objects that will respect the
circle multiplication $ r \cdot s = r + s + rs $, and to compute adjoint semigroups and adjoint groups of finite rings. Also
it may serve as an example of extending the \textsf{GAP} system with new multiplicative objects. 

 \mbox{}}\\[1cm]
{\small 
\section*{Copyright}
\logpage{[ 0, 0, 2 ]}
 {\copyright} 2006-2024 by Alexander Konovalov and Panagiotis Soules 

 \textsf{Circle} is free software; you can redistribute it and/or modify it under the terms of
the GNU General Public License as published by the Free Software Foundation;
either version 2 of the License, or (at your option) any later version. For
details, see the FSF's own site \href{http://www.gnu.org/licenses/gpl.html} {\texttt{http://www.gnu.org/licenses/gpl.html}}. 

 If you obtained \textsf{Circle}, we would be grateful for a short notification sent to one of the authors. 

 If you publish a result which was partially obtained with the usage of \textsf{Circle}, please cite it in the following form: 

 A. Konovalov, P. Soules. \emph{Circle --- Adjoint groups of finite rings, Version 1.0;} 2024 (\href{http://www.cs.st-andrews.ac.uk/~alexk/circle/} {\texttt{http://www.cs.st-andrews.ac.uk/\texttt{\symbol{126}}alexk/circle/}}). \mbox{}}\\[1cm]
{\small 
\section*{Acknowledgements}
\logpage{[ 0, 0, 3 ]}
 We acknowledge very much Alexander Hulpke and James Mitchell for their helpful
comments and advices, and the referee for testing the package and useful
suggestions. 

 \mbox{}}\\[1cm]
\newpage

\def\contentsname{Contents\logpage{[ 0, 0, 4 ]}}

\tableofcontents
\newpage

 
\chapter{\textcolor{Chapter }{Introduction}}\label{Intro}
\logpage{[ 1, 0, 0 ]}
\hyperdef{L}{X7DFB63A97E67C0A1}{}
{
  
\section{\textcolor{Chapter }{General aims}}\label{IntroAbstract}
\logpage{[ 1, 1, 0 ]}
\hyperdef{L}{X8557083378F2A3B2}{}
{
  Let $R$ be an associative ring, not necessarily with one. The set of all elements of $R$ forms a monoid with the neutral element $0$ from $R$ under the operation $ r \cdot s = r + s + rs $ defined for all $r$ and $s$ of $R$. This operation is called the \emph{circle multiplication}, and it is also known as the \emph{star multiplication}. The monoid of elements of $R$ under the circle multiplication is called the adjoint semigroup of $R$ and is denoted by $R^{ad}$. The group of all invertible elements of this monoid is called the adjoint
group of $R$ and is denoted by $R^{*}$. 

 These notions naturally lead to a number of questions about the connection
between a ring and its adjoint group, for example, how the ring properties
will determine properties of the adjoint group; which groups can appear as
adjoint groups of rings; which rings can have adjoint groups with prescribed
properties, etc. 

 For example, V. O. Gorlov in \cite{Gorlov-1995} gives a full list of finite nilpotent algebras $R$, such that $R^2 \ne 0$ and the adjoint group of $R$ is metacyclic (but not cyclic). 

 S. V. Popovich and Ya. P. Sysak in \cite{Popovich-Sysak-1997} characterize all quasiregular algebras such that all subgroups of their
adjoint group are their subalgebras. In particular, they show that all
algebras of such type are nilpotent with nilpotency index at most three. 

 Various connections between properties of a ring and its adjoint group were
considered by O. D. Artemovych and Yu. B. Ishchuk in \cite{Artemovych-Ishchuk-1997}. 

 B. Amberg and L. S. Kazarin in \cite{Amberg-Kazarin-2000} give the description of all nonisomorphic finite $p$-groups that can occur as the adjoint group of some nilpotent $p$-algebra of the dimension at most 5. 

 In \cite{Amberg-Sysak-2001} B. Amberg and Ya. P. Sysak give a survey of results on adjoint groups of
radical rings, including such topics as subgroups of the adjoint group;
nilpotent groups which are isomorphic to the adjoint group of some radical
ring; adjoint groups of finite nilpotent \$p\$-algebras. The authors continued
their investigations in further papers \cite{Amberg-Sysak-2002} and \cite{Amberg-Sysak-2004}. 

 In \cite{Kazarin-Soules-2004} L. S. Kazarin and P. Soules study associative nilpotent algebras over a field
of positive characteristic whose adjoint group has a small number of
generators. 

 The main objective of the proposed \textsf{GAP}4 package \textsf{Circle} is to extend the \textsf{GAP} functionality for computations in adjoint groups of associative rings to make
it possible to use the \textsf{GAP} system for the investigation of the above described questions. 

 \textsf{Circle} provides functionality to construct circle objects that will respect the
circle multiplication $ r \cdot s = r + s + rs $, create multiplicative structures, generated by such objects, and compute
adjoint semigroups and adjoint groups of finite rings. 

 Also we hope that the package will be useful as an example of extending the \textsf{GAP} system with new multiplicative objects. Relevant details are explained in the
next chapter of the manual. }

  
\section{\textcolor{Chapter }{Installation and system requirements}}\label{IntroInstall}
\logpage{[ 1, 2, 0 ]}
\hyperdef{L}{X7DB566D5785B7DBC}{}
{
  \textsf{Circle} does not use external binaries and, therefore, works without restrictions on
the type of the operating system. This version of the package is designed for \textsf{GAP}4.5 and no compatibility with previous releases of \textsf{GAP}4 is guaranteed. 

 To use the \textsf{Circle} online help it is necessary to install the \textsf{GAP}4 package \textsf{GAPDoc} by Frank L{\"u}beck and Max Neunh{\"o}ffer, which is available from the \textsf{GAP} site or from \href{http://www.math.rwth-aachen.de/~Frank.Luebeck/GAPDoc/} {\texttt{http://www.math.rwth-aachen.de/\texttt{\symbol{126}}Frank.Luebeck/GAPDoc/}}. 

 \textsf{Circle} is distributed in standard formats (\texttt{tar.gz}, \texttt{tar.bz2}, \texttt{zip} and \texttt{-win.zip}) and can be obtained from \href{http://www.cs.st-andrews.ac.uk/~alexk/circle/} {\texttt{http://www.cs.st-andrews.ac.uk/\texttt{\symbol{126}}alexk/circle/}} or from the \textsf{GAP} homepage. To install the package, unpack its archive in the \texttt{pkg} subdirectory of your \textsf{GAP} installation. }

 }

 \def\bibname{References\logpage{[ "Bib", 0, 0 ]}
\hyperdef{L}{X7A6F98FD85F02BFE}{}
}

\bibliographystyle{alpha}
\bibliography{manual}

\addcontentsline{toc}{chapter}{References}

\def\indexname{Index\logpage{[ "Ind", 0, 0 ]}
\hyperdef{L}{X83A0356F839C696F}{}
}

\cleardoublepage
\phantomsection
\addcontentsline{toc}{chapter}{Index}


\printindex

\newpage
\immediate\write\pagenrlog{["End"], \arabic{page}];}
\immediate\closeout\pagenrlog
\end{document}
